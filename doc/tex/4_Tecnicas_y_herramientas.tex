\capitulo{4}{Técnicas y herramientas}


\section{Entorno de desarrollo}

En este apartado se describen las principales técnicas y herramientas utilizadas para el desarrollo del proyecto. 


\subsection{Eclipse}

\href{https://www.eclipse.org/ide/}{Eclipse} es un entorno de desarrollo integrado (IDE) ampliamente utilizado para programar aplicaciones en Java y 
otros lenguajes de programación. Eclipse permite gestionar proyectos Java, editar código con resaltado de sintaxis,
 depurar en tiempo real, etc. 
\

Los nodos de KNIME se desarrollan en Java utilizando el IDE de programación Eclipse. KNIME nos facilita instrucciones de 
instalaciónd el entorno de desarrollo con Eclipse para desarrollar nuestros nodos personalizados. 


\subsection{KNIME SDK}


\href{https://github.com/knime/knime-sdk-setup}{KNIME SDK} (Software Development Kit) es el conjunto de herramientas facilitadas por KNIME que nos permiten desarrollar 
extensiones y nodos personalizados que pueden intergrarse en los flujos de trabajo de KNIME.
\

KNIME SDK se integra con Eclipse para desarrollar componentes de KNIME. 


\subsection{Bitnami LMS Virtual Machine}

\href{https://bitnami.com/}{Bitnami} es una empresa especializada en la creación de stacks de aplicaciones, 
que son entornos de aplicaciones preconfigurados para facilitar la instalación y puesta en marcha de una aplicación junto 
con todas sus dependencias. 
\

\href{https://bitnami.com/stack/moodle}{Bitnami LMS} es una distribución que incluye el LMS Moodle y todas las aplicaciones 
necesarias para su ejecución (PHP, MySQL, servidor web, etc.). La distribución está disponible como instalación en la nube 
o como instalación en local con Docker, Kubernetes o Máquina Virtual. 
\

En este proyecto se ha utilizado la versión en Máquina Virtual (formato OVA) correspondiente a Moodle 3.11. 
\

\subsection{VirtualBox}

\href{https://www.virtualbox.org/}{VirtualBox} es un software de virtualización de código abierto que permite ejecutar máquinas 
virtuales en cualquier sistema operativo. En este proyecto se ha utilizado VirtualBox 6.1 para ejectuar la máquina virtual 
facilitada por Bitnami LMS y disponer así de una instancia completamente funcional de Moodle en el equipo local. 


\subsection{Insomnia Rest}

\href{https://insomnia.rest/}{Insomnia Rest} es una aplicación de escritorio que nos permite realizar llamadas REST a APIs de
 aplicaciones basadas en este sistema de comunicación. Una particularidad de Insomnia Rest es que permite realizar 
 pruebas sobre aplicaciones en local, lo que nos ha permitido probar la comunicación con 
 la API de Moodle de nuestra máquina virtual local, antes de incorporar el código a los nodos programados de KNIME.

\subsection{GitHub}

\href{https://github.com/}{GitHub} es una plataforma de desarrollo de software en la nube que utiliza Git como sistema de control 
de versiones. GitHub permite a los desarroladores colaborar en proyectos de programación, alojando y gestionando sus repositorios de 
código fuente. En este proyecto se ha usado GitHub para almacenar el código fuente tanto de la aplicación desarrollada como de la memoria. 

\newpage
\section{Desarrollo}

En este apartado se describen los lenguajes de programación y librerías utilizados para el desarrollo del proyecto. 

\subsection{Java}

Aunque desde la versión 4.6 de KNIME también se permite la implementación de nodos en Python, en este proyecto los nodos se 
han implementado en Java. 
\

\href{https://www.java.com/}{Java} es un lenguaje de programación muy extendido que se caracteriza por su programación orientada a objetos y su portabilidad. Las 
aplicaciones JAVA se compilan y pueden ser utilizadas en cualquier plataforma o sistema operativo que disponga de la Máquina 
Virtual Java (JVM).

\subsection{Moodle API}

Moodle cuenta con una REST API que permite la comunicación con el sistema a través de servicios web. El listado completo de servicios
web de Moodle puede consultarse en este enlace (\href{https://docs.moodle.org/dev/Web_service_API_functions}{Moodle API}). 


\subsection{KNIME API}

El código del núcleo de KNIME y de otras extensiones está disponible en el \href{https://github.com/knime/}{repositorio de KNIME en GitHub} para poder ser reutilizado en nuevos desarrollos de extensiones
de KNIME. 

\subsection{Librerías externas}

Algunas librerías externas que se han incorporado al proyecto son: 

\subsubsection{UBUMonitor}

\href{https://github.com/yjx0003/UBUMonitor}{UBUMonitor} es una herramienta implementada en Java que permite que los usuarios con 
rol de profesor puedan monitorización a sus alumnos en plataformas LSM Moodle. 

\subsubsection{Data Faker}

\href{https://www.datafaker.net/}{Data Faker} es una librería para Java que permite crear datos falsos para pruebas en aplicaciones JVM. 



\newpage
\section{Memoria}

En el desarrollo de la Memoria del proyecto se han utilizado las siguientes herramientas: 

\subsection{Visual Studio Code}

\href{https://code.visualstudio.com/}{Visual Studio Code}, también conocido como VS Code, es un editor de código fuente gratuito y de código abierto
 desarrollado por Microsoft. VS Code se puede extender fácilmente instalando extensiones adicionales, lo que permite que pueda ser utilizado para una
  amplia variedad de lenguajes de programación. En este proyecto se ha utilizado VS Code para trabajar con LaTeX.

\subsection{LaTeX}

\href{https://www.latex-project.org/}{LaTeX} es un sistema que permite crear documentos estructurados a partir de comandos y etiquetas. 
LaTeX produce documentos con un formato profesional, lo que lo hace ideal para desarrollar tesis, artículos académicos, informes técnicos, etc.

