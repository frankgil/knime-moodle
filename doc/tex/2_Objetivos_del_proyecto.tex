\capitulo{2}{Objetivos del proyecto}
En este apartado se exponen los objetivos del proyecto, diferenciados entre objetivos generales, técnicos y personales. 

\section{Objetivos generales}

\begin{itemize}
	\item Desarrollar un plugin de Knime que permita importar datos desde una plataforma 
	de teleformación Moodle, para su tratamiento posterior dentro de flujos de trabajo de Knime.
	\item Facilitar la incorporación en Knime de datos de cursos de cualquier plataforma Moodle, 
	sin necesidad de realizar ajustes personalizados dentro de la plataforma de formación. 
    \item Orientar la solución al perfil de Profesor, de forma que cualquier usuario con este
	 perfil pueda realizar estudios desde Knime de los datos relativos a sus cursos. 
    \item Implementar un ejemplo práctico utilizando el plugin de Knime desarrollado en un
	 flujo de trabajo relacionado con Aprendizaje Supervisado.
\end{itemize}

\section{Objetivos técnicos}

\begin{itemize}
	\item Desarrollar un plugin para Knime en lenguaje de programación Java. 
	\item Explorar la arquitectura de Knime para elegir el tipo de componentes a desarrollar que mejor se adapten a la solución requerida.
	\item Conocer a fondo la metodología de programación en Knime mediante el estudio de la documentación existente y la inspección de plugins de Knime similares.
	\item Explorar los nodos y workflows de Knime disponibles en el ámbito del Aprendizaje Supervisado.
\end{itemize}

\section{Objetivos personales}

\begin{itemize}
	\item Aplicar los conocimientos adquiridos dentro del Máster en al ámbito educativo y más específicamente en el de la formación online. 
	\item Conocer más a fondo las herramientas que, como Knime, permiten el análisis de datos desde la interfaz de usuario, tanto a nivel de usuario como a nivel de desarrollador. 
    \item Contribuir a la mejora de la formación online al facilitar que los profesores puedan realizar análisis sobre sus cursos sin conocimientos de programación. 
\end{itemize}
