\capitulo{7}{Conclusiones y Líneas de trabajo futuras}

Todo proyecto debe incluir las conclusiones que se derivan de su desarrollo. Éstas pueden ser de diferente índole, dependiendo de la tipología del proyecto, pero normalmente van a estar presentes un conjunto de conclusiones relacionadas con los resultados del proyecto y un conjunto de conclusiones técnicas. 
Además, resulta muy útil realizar un informe crítico indicando cómo se puede mejorar el proyecto, o cómo se puede continuar trabajando en la línea del proyecto realizado. 


Integración con Moodle a través de web services. 
Permitir una integración que permita añadir estructuras más complejas... como analizar un grupo de cursos. 
Esta línea requiere no pensar solo en el rol de profesor, sino en la intervención por parte de la institución para facilitar la integración y el acceso a los datos


Colección de Flujos de trabajo típicos con Knime. Knime permite crear flujos de trabajo y compartirlos, de forma que sea fácil su reutilización. 


Knime en cloud o servidor para flujos de trabajo "permanentes" (cron). Lo que hemos visto se basa solo en ejecución de knime desde local. 
Extender a flujos 

Comunicación bidireccional con Moodle. Tomar acciones a partir de los resultados del procesamiento de los datos. Por ejemplo, enviar un correo a los alumnos clasificados como en riesgo de abandono. 
