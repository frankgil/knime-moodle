\capitulo{7}{Conclusiones y Líneas de trabajo futuras}


\section{Conclusiones}

Los objetivos generales planteados al inicio del proyecto se han podido alcanzar a la finalización del mismo, con 
lo que se ha conseguido desarrollar una extensión de KNIME muy completa que permite extraer datos de Moodle e
 incorporarlos en flujos de trabajo más complejos para estudios relacionados con la Ciencia de Datos. 
\ 

A nivel técnico se ha requerido un profundo estudio del lenguaje de programación Java, no solo orientado al desarrollo
específico de extensiones de KNIME, sino más amplio por haberse requeridointegración con otras aplicaciones externas. 
También se ha ahondado en la arquitectura de KNIME, tanto a nivel de núcleo como a nivel de estructura de extensiones y nodos. 

También se han podido cumplir los objetivos personales planteados, ampliando los conocimientos adquiridos en el Máster y 
aplicándolos a nuevas herramientas como KNIME y a datos de estudio de mi interés personal y de investigación, relacionados 
con la formación online. 

Sin duda, considero que se ha realizado un trabajo amplio y con mucho interés personal en la materia de estudio. 
Gracias a una tutorización continua por parte de los tutores, el proyecto ha estado muy bien organizado desde 
su inicio y se ha podido entregar una extensión de KNIME totalmente funcional, pero al mismo tiempo abierta para servir 
de base para futuras ampliaciones. 


\section{Líneas de trabajo futuras}

Debido al alcance limitado del proyecto, se han quedado fuera algunas posibles mejoras y ampliaciones, que se exponen a continuación: 

\subsection{Desarrollo de plugins de Moodle para extraer información adicional}

En este proyecto solo se ha contemplado la extracción de datos a través de las herramientas que facilita Moodle. 
Una integración más ambiciosa entre KNIME y Moodle podría conllevar el desarrollo de plugins de Moodle para extraer información más específica que no se puede extraer por defecto. 

Esta línea requiere no pensar únicamente en el rol de profesor, sino en la intervención por parte de las instituciones educativas para facilitar la integración 
y el acceso a datos adicionales. 


\subsection{Colección de Flujos de trabajo de KNIME para Moodle}

Knime permite crear flujos de trabajo y compartirlos, de forma que sea fácil su reutilización. Se propone en esta línea crear un set de 
workflows ya preparados para su utilización con la extensión de Moodle desarrollada. 


\subsection{Knime en servidor}

Los flujos de trabajo estudiados en este proyecto se basan en la ejecución local de una instancia de KNIME. La propuesta es utilizar 
KNIME en cloud o servidor para flujos de trabajo que actúan de forma permanente. Por ejemplo, un flujo que revisa constantemente la evolución de un curso para 
determinar si algún alumno está en riesgo de abandono. 

\subsection{Comunicación bidireccional entre KNIME y Moodle}

En este proyecto solo se ha contemplado la extracción de datos desde Moodle hacia KNIME. Moodle permite, a través de determinados 
servicios web ya incluidos o a través de servicios web personalizados, realizar acciones en el aula. Sería interesante que el resultado 
de la ejecución de un workflow de KNIME pudiera realizar acciones a partir de los resultados obtenidos. Por ejemplo, enviar un correo a
 los alumnos clasificados como en riesgo de abandono o publicar un mensaje de aviso en el foro del aula. 
 
\subsection{Datos Abiertos (Open Data)}

Conseguir que las instituciones educativas involucradas en el proyecto faciliten datos abiertos de forma automática con una política previa de anonimización (Open Data). 
Aunque existen intentos de publicar información basados en la información extrictamente legal (portal de transparencia) y datos preprocesados 
con estadísticas finales, con el auge de la Ciencia de Datos, se hace cada vez más necesario el acceso a datos en bruto para análisis más variados. 

\subsection{Anonimización}

Realizar una anonimización ajustada a la política de anonimización de la institución y que cubra la anonimización de identificadores de usuarios. 


\subsection{Obtener género}

Mejorar la API incluyendo varios proveedores externos. 


\subsection{Desvincular de librerías externas}

Eliminar las dependencias con las librerías externas. Por ejemplo, eliminar el código dependiente de UBUMonitor desarrollando soluciones propias integradas en la extensión. 

\subsection{Actualización}

Actualización a la última versión de KNIME, actualmente la 5.x. Definir un plan de mantenimiento y actualizaciones a versiones futuras. 


\subsection{Ampliación de funcionalidad}

Ampliar funcionalidad extrayendo más información disponible en Moodle. Por ejemplo, la versión actual está limitada en los tipos de logs que se extraen. 


\subsection{Ampliar pruebas}

Las pruebas que se han llevado a cabo han estado muy dirigidas hacia un tipo de cursos y una plataforma Moodle concreta. Se considera necesario 
ampliar las pruebas a otros entornos de Moodle y otros tipos de acciones formativas. 


\subsection{Compartir extensión}

Compartir la extensión en el Community HUB para que esté disponible para otros usuarios de la comunidad. 


