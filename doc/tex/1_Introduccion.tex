\capitulo{1}{Introducción}

Moodle es una plataforma de aprendizaje (LMS, Learning Management System) ampliamente utilizada por instituciones educativas de todo el mundo, incluyendo universidades. 
Durante el tiempo de vida de una acción formativa o curso, Moodle registra todo tipo de información relacionada con la interacción de los participantes con la plataforma.  Por ejemplo, 
 se registra cada vez que un usuario consulta un recurso, visita una página, visualiza un vídeo o participa en un foro de discusión. También quedan registradas las 
 calificaciones obtenidas en las diferentes actividades evaluables y los tiempos empleados en la realización de cuestionarios. Esta información puede ser muy variada, 
 ya que depende de la configuración de los componentes integrados dentro de cada curso. Ahora bien, aunque mucha de esta información registrada puede ser consultada por el profesor a través de informes que están disponibles dentro del curso, no es fácilmente exportable 
 para su procesamiento desde herramientas externas. 
\

Utilizando técnicas de aprendizaje automático se puede extraer conocimiento sobre el curso y sus participantes. Este conocimiento, en manos del profesor responsable, puede servir para mejorar el curso 
y/o tomar decisiones que afecten al desempeño de los estudiantes. Supongamos, por ejemplo, que el profesor dispone de un modelo que le permite predecir qué estudiantes están en riesgo de abandonar 
o no finalizar el curso. A partir de esta información, el profesor puede tomar las medidas que considere necesarias para intentar ``recuperar'' a estos 
estudiantes (enviar un correo electrónico, agendar una tutoría individual, proponer materiales de apoyo, etc.). Además, se podría volver a aplicar el mismo modelo pasado un tiempo para comprobar si los
 estudiantes han salido de la zona de riesgo o, por el contrario, se requieren acciones adicionales. 
\

En las últimas versiones de Moodle se han ido incorporando algunos modelos predictivos, tanto en el núcleo como a través de plugins externos. Estos modelos pueden ser útiles para los casos de uso más comunes,
pero siempre estarán limitados con respecto a las posibilidades de analizar los datos de una forma más libre y abierta desde herramientas externas, como es el caso de la herramienta KNIME que abordamos en este trabajo.
\

KNIME es una plataforma de código abierto orientada al análisis de datos que permite, a través de una interfaz intuitiva y visual, 
crear análisis complejos y flujos de trabajo mediante componentes especializados en diferentes áreas como la minería de datos, 
el aprendizaje automático o la visualización de datos. En este trabajo se han desarrollado nuevos componentes de KNIME que permiten la 
integración en sus flujos de trabajo de los datos que Moodle va registrando durante la ejecución de una acción formativa. 
Con un enfoque orientado al rol de profesor, se pretende que cualquier usuario con este rol dentro de una 
plataforma Moodle, pueda realizar estudios externos sobre los datos de las acciones formativas a las que tiene acceso. 
Adicionalmente se ha implementado un flujo de trabajo en KNIME sobre datos reales extraidos de Moodle, analizando diferentes 
técnicas de aprendizaje automático supervisado. 
\


\section{Estructura de la memoria}

La memoria se compone de los siguientes apartados: 
\

\begin{enumerate}
	\item \textbf{Introducción}. Introducción al proyecto y al trabajo desarrollado. 
	\item \textbf{Objetivos del proyecto}. Se describen los objetivos generales, técnicos y personales establecidos al inicio del proyecto.
	\item \textbf{Conceptos teóricos}. Se introducen los principales conceptos teóricos abordados en el proyecto. 
	\item \textbf{Técnicas y herramientas}. Se describen las herramientas utilizadas para el desarrollo del proyecto.
	\item \textbf{Aspectos relevantes del desarrollo}. Recoge los aspectos más interesantes y relevantes del desarrollo del proyecto.
	\item \textbf{Trabajos relacionados}. Se describen otros trabajos relacionados con el proyecto y el grado de influencia que han tenido en el desarrollo.
	\item \textbf{Conclusiones y líneas de trabajo futuras}. Conclusiones finales del proyecto y posibles líneas de trabajo que se podrían seguir para ampliar o dar continuidad al proyecto. 
\end{enumerate}


\section{Estructura de los apéndices}

También se incluyen los siguientes apéndices anexos al proyecto: 
\

\begin{enumerate}[A.]
	\item \textbf{Plan de proyecto}. Descripción de la metodología utilizada, planificación temporal y viabilidad económica. 
	\item \textbf{Especificación de diseño}. Incluye la especificación de requisitos y el diseño implementado.
	\item \textbf{Documentación técnica de programación}. Descripción de los pasos a seguir para instalar, compilar y ejecutar el proyecto.
	\item \textbf{Documentación de usuario}. Manual de usuario de los componentes KNIME desarrollados. 
\end{enumerate}

Tanto la memoria como el código desarrollado se puede consultar y descargar desde el \href{https://github.com/frankgil/knime-moodle}{repositorio GitHub del proyecto}.