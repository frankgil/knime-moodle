\capitulo{6}{Trabajos relacionados}

\section{UBU Monitor}

UBUMonitor es una aplicación de escritorio que permite la conexión a una plataforma Moodle para monitorizar la actividad
de los alumnos. UBUMonitor permite extraer los datos de los logs y las calificaciones obtenidas en las actividades, presentando 
los datos con herramientas de visualización que lo hacen muy atractivo. Además, añade modelos de aprendizaje automático 
para monitorizar el riesgo de abandono de los alumnos matriculados \cite{ubumonitor}. 
\

Nuestro proyecto comparte mucho con UBUMonitor en cuando al tipo de información con la que trabaja y perfil de usuario al que
 va destinado, y nos ha resultado muy útil como referencia e incluso reutilización de algunos componentes. 
\

Sin embargo, nuestra aproximación utilizando el ecosistema de KNIME, intenta ser más abierta y flexible, dejando que sea 
el usuario final el que pueda decidir qué tipo de modelos implementar o utilizar con los datos extraídos de sus asignaturas. 


