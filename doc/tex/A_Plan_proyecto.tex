\apendice{Plan de Proyecto Software}

\section{Metodología}

Durante el desarrollo de este proyecto se ha utilizado una versión simplificada de la metodología Scrum. Concretamente se han
seguido ciclos o sprints de dos semanas, apoyados por reuniones de coordinación con los tutores del proyecto. Tras cada sprint se 
han mostrado y discutido los avances realizados y se han planificado las tareas a realizar en el siguiente sprint. 
\

Aunque durante el desarrollo del proyecto se han producido retrasos que han llevado a dilatar la 
entrega final del proyecto, el orden de ejecución de los sprints y el alcance del proyecto se han 
mantenido según la planificación inicial, que se detalla a continuación. 


\section{Planificación temporal}

En este apartado se describen las tareas llevadas a cabo en cada sprint. 

\subsection{Sprint 0}

Sprint de arranque del proyecto dedicado al análisis del problema y el alcance de la solución. 

\begin{itemize}
	\item Preparación del entorno de desarrollo para programar extensiones de Knime.
	\item Desarrollo de nodo de prueba para conocer la arquitetura de Knime.
\end{itemize}


\subsection{Sprint 1}

En este sprint continuamos con la fase de Análisis: 

\begin{itemize}
	\item Preparación de plataforma Moodle local con datos de prueba. 
	\item Pruebas de acceso a la información de Moodle a través de web services y web scraping. 
    \item Estudio de la aplicación UBUMonitor, tanto a nivel de usuario como a nivel de código, para evaluar su posible reutilización.
    \item Se define el alcance y objetivos del proyecto.
    \item Memoria: se prepara el documento base de Memoria y se completan los apartados de Introducción, Objetivos y Análisis. 
\end{itemize}


\subsection{Sprint 2}

Sprint dedicado al diseño de algunos de los módulos propuestos durante la fase de análisis. 

\begin{itemize}
	\item Diseño de nodo Moodle Connection.
	\item Diseño de nodo Moodle Courses.
	\item Diseño de nodo Moodle Users.
	\item Memoria: documentación parcial de los apartados de Técnicas y herramientas y Diseño.
\end{itemize}


\subsection{Sprint 3}

Implementación parcial de los nodos diseñados en el sprint anterior. En este sprint los nodos implementados son operativos 
pero no se ha añadido aún configuración. 

\begin{itemize}
	\item Implementación parcial del nodo Moodle Connection.
	\item Implementación parcial del nodo Moodle Courses.
	\item Implementación parcial del nodo Moodle Users.
	\item Workflows básicos para probar los nodos implementados con datos de prueba de Moodle. 
	\item Memoria: documentación parcial del apartado de Implementación.
\end{itemize}


\subsection{Sprint 4}

Ampliación de los nodos implementados en el sprint anterior añadiendo configuración. 

\begin{itemize}
	\item Implementación de configuración del nodo Moodle Connection.
	\item Implementación de configuración del nodo Moodle Courses.
	\item Implementación de configuración del nodo Moodle Users.
	\item Memoria: actualización del apartado de Implementación. 
\end{itemize}


\subsection{Sprint 5}

Sprint dedicado al diseño de otros módulos propuestos durante la fase de análisis. 

\begin{itemize}
	\item Diseño del nodo Moodle Reports Logs y primeras pruebas de implementación. 
	\item Memoria: actualización del apartado de Diseño.
\end{itemize}



\subsection{Sprint 6}

Sprint dedicado a la implementación del nodo Moodle Reports Logs. 

\begin{itemize}
	\item Implementación del nodo Moodle Reports Logs. 
	\item Mapeo de logs de Moodle para extraer variables de los textos de log. 
	\item Memoria: actualización del apartado de Implementación. 
\end{itemize}



\subsection{Sprint 7}

Sprint dedicado al diseño de otros módulos propuestos durante la fase de análisis. 

\begin{itemize}
	\item Diseño del nodo Moodle Reports Grades y primeras pruebas de implementación. 
	\item Memoria: actualización del apartado de Diseño.
\end{itemize}


\subsection{Sprint 8}

Sprint dedicado a la implementación del nodo Moodle Reports Grades. 

\begin{itemize}
	\item Implementación del nodo Moodle Reports Grades. 
	\item Memoria: actualización del apartado de Implementación. 
\end{itemize}


\subsection{Sprint 9}

Sprint dedicado a la ampliación del nodo Moodle Users. 

\begin{itemize}
	\item Incorporación de Anonimización de usuarios.
	\item Estimación de género de usuario. 
	\item Memoria: actualización del apartado de Implementación. 
\end{itemize}



\subsection{Sprint 10}

Sprint dedicado a la preparación final del código y compilación de la solución final. 

\begin{itemize}
	\item Limpieza del código eliminando mensajes de depuración y añadiendo comentarios.
	\item Compilación final para generar el plugin que puede ser utilizado en cualquier workflow de Knime. 
	\item Memoria: actualización del apartado de Implementación. 
\end{itemize}



\subsection{Sprint 11}

Primer sprint dedicado al uso de los nodos implementados con datos reales. 

\begin{itemize}
	\item Importación de datos reales en Moodle local. 
	\item Implementación de workflows básicos para probar el correcto acceso a datos reales. 
	\item Memoria: actualización del apartado de Workflows. 
\end{itemize}




\subsection{Sprint 12}

Sprint dedicado a la implementación del Workflow con datos reales. 

\begin{itemize}
	\item Implementación del worflow con datos reales. 
	\item Memoria: actualización del apartado de Workflows. 
\end{itemize}



\subsection{Sprint 13}

Sprint dedicado a la implementación del Workflow con datos reales (continuación). 

\begin{itemize}
	\item Implementación del worflow con datos reales. 
	\item Memoria: actualización del apartado de Workflows. 
	\item Memoria: revisión general. 
\end{itemize}


\subsection{Sprint 14}

Sprint dedicado a completar la memoria del proyecto. 

\begin{itemize}
	\item Memoria: actualización de apartados pendientes
	\item Memoria: corrección general según sugerencias reportadas en el sprint anterior. 
\end{itemize}

\newpage
\section{Estudio de viabilidad}

\subsection{Viabilidad económica}
\subsubsection{Costes de personal}
Aunque el proyecto se ha estimado en 7 meses a tiempo parcial, para calcular el coste del proyecto, se 
considerará que se ha realizado por un desarrollador a tiempo completo en un período de 3,5 meses. 
Se considera el salario medio neto de un programador de 1.610€ mensuales \cite{salario}.
(ver tabla~\ref{tab:personal}).

La cotización a la seguridad social se ha calculado como horas comunes, según
el régimen general de 2023 (28,30\%)~\cite{seguridad-social}.

\begin{table}[!h]
	\centering
	\begin{tabular}{lr}
		\toprule
		\textbf{Concepto} & \textbf{Coste} \\
		\midrule
		Salario neto & 1610,00€\\
		Retención IRPF (17\%) & 500,37€ \\
		Seguridad social (28,30 \%) & 832,96€ \\
		\midrule
		Salario bruto (mensual) & 2.943,33€ \\
		\midrule
		\textbf{Total 3.5 meses} & 10.301,65 \\
		\bottomrule
	\end{tabular}
	\caption{Costes de personal}
	\label{tab:personal}
\end{table}

Además se sumará el sueldo de los dos tutores asignados al proyecto \cite{misc:retribuciontutores-funcionarios} durante 7 meses, ambos con la posición de Profesor Titular de Universidad. Se asignan 0,5 por tutor.

Sueldo mensual: 3.145,59€. Imparte 24 créditos anuales.
$$ \dfrac{\textup{3.145,59€} \times 12 \;meses}{24 
	\;créditos} \times 0,5 \;créditos \times 2 \;tutores = \textup{1.572,80€} $$


El coste de los tutores corresponde a 1.572,80€.

\subsubsection{Costes de \textit{hardware}}
En esta sección se enumeran los costes del \textit{hardware} usado durante el desarrollo.

Para el desarrollo se ha usado un equipo de sobremesa valorado en 1200€, con 
amortización en 4 años (ver tabla~\ref{tab:hardware}).

$$\dfrac{\textup{1200€}}{4 \;años * 12 \;meses} = 
\textup{25} $$

\begin{table}[!h]
	\centering
	\begin{tabular}{lrr}
		\toprule
		\textbf{Concepto} & \textbf{Coste} & \textbf{Amortización} \\
		\midrule
		Ordenador sobremesa & 1200€ & 25 \\
		\midrule
		\textbf{Total 3,5 meses} & 87,5€ \\
		\bottomrule
	\end{tabular}
	\caption{Costes de hardware}
	\label{tab:hardware}
\end{table}

\subsubsection{Costes totales}

En la tabla~\ref{tab:total} se agrupan todos los costes calculados del proyecto, dando el total de €.

\begin{table}[!h]
	\centering
	\begin{tabular}{lr}
		\toprule
		\textbf{Concepto} & \textbf{Coste} \\
		\midrule
		Personal & 10.301,65€ \\
		Tutores  & 1.572,80€ \\
		\textit{Hardware} & 87,51€ \\
		\midrule
		\textbf{Total} & 11.961,96€ \\
		\bottomrule
	\end{tabular}
	\caption{Costes totales del proyecto}
	\label{tab:total}
\end{table}

\subsection{Viabilidad legal}

KNIME es una herramienta de software libre que puede ser descargada y utilizada gratuitamente bajo los 
términos de la Licencia Pública General de GNU versión 3 (GPLv3). Hay que tener en cuenta que, aunque la 
plataforma es gratuita, algunas extensiones de terceros pueden tener licencias específicas y 
restrictiones de uso. En este proyecto se han utilizado extensiones adicionales de KNIME para 
definir los workflows de trabajo, pero no se han requerido estas extensiones dentro del código 
de la extensión y nodos desarrollados, por lo que solo se deben tener en cuenta los términos de la
 licencia GPLv3 del núcleo de KNIME. 

Adicionalmente se ha incluido código de la aplicación externa UBUMonitor, distribuido bajo licencia MIT. Esta licencia permite, entre 
otras cosas, la modificación, distribución y uso comercial del código fuente, por lo que el código 
de UBUMonitor se puede incorporar a nuestro proyecto sin restricciones, pero a su vez sin garantías de ningún tipo. 

