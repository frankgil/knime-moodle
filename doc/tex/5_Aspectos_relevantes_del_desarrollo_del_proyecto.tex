\capitulo{5}{Aspectos relevantes del desarrollo del proyecto}

Este apartado pretende recoger los aspectos más interesantes del desarrollo del proyecto, comentados por los autores del mismo.
Debe incluir desde la exposición del ciclo de vida utilizado, hasta los detalles de mayor relevancia de las fases de análisis, diseño e implementación.
Se busca que no sea una mera operación de copiar y pegar diagramas y extractos del código fuente, sino que realmente se justifiquen los caminos de solución que se han tomado, especialmente aquellos que no sean triviales.
Puede ser el lugar más adecuado para documentar los aspectos más interesantes del diseño y de la implementación, con un mayor hincapié en aspectos tales como el tipo de arquitectura elegido, los índices de las tablas de la base de datos, normalización y desnormalización, distribución en ficheros3, reglas de negocio dentro de las bases de datos (EDVHV GH GDWRV DFWLYDV), aspectos de desarrollo relacionados con el WWW...
Este apartado, debe convertirse en el resumen de la experiencia práctica del proyecto, y por sí mismo justifica que la memoria se convierta en un documento útil, fuente de referencia para los autores, los tutores y futuros alumnos.



\section{Soluciones de acceso a datos en Moodle}

Comentar aquí solución web services, limitaciones tanto de acceso a datos como de 

Finalmente se opta por web scraping utilizando la cuenta del profesor para acceder a los logs. 

Login con webservice de la APP de Moodle. Esto implica que la implementación de Moodle debe tener activa la opción de acceso de la APP. 



\section{Desarrollo de nodos en Knime}


Extensión: 

- KNIME Moodle Integration


Nodos desarrollados:

- Moodle Connector. 

Establece la conexión con una plataforma Moodle. Requiere una cuenta con perfil de profesor. 
Requiere que la plataforma Moodle tenga activado el acceso a la aplicación Móvil. 

Output Ports. 

0. A connection that can be used to access Moodle's API.

URL
usuario
contraseña


Salida: token

- Moodle Courses. 

core\_course\_get\_courses
core\_course\_get\_courses\_by\_field
core\_course\_search\_courses

Input Ports: 

0 Connection to Moodle's API (token)
1 Table containing... 

Output Ports: 

0 Información de cursos (ID cursos)



- Moodle Users. 

core\_enrol\_get\_enrolled\_users
anonymization


- Moodle Reports

logs
Live logs
Activity Reports
Overview Statistics
Course Participation
Activity completion
Statistics


Workflows desarrollados: 

Nota: los workflows compartidos son casi tan importantes como los nodos desarrollados, ya que permiten el uso directo de soluciones 

- Workflow básico (extraer logs)
- Workflow clasificación
- Workflow clustering


\section{Acceso a datos en Moodle}
