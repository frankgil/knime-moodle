\capitulo{5}{Aspectos relevantes del desarrollo del proyecto}

Este apartado pretende recoger los aspectos más interesantes del desarrollo del proyecto, comentados por los autores del mismo.
Debe incluir desde la exposición del ciclo de vida utilizado, hasta los detalles de mayor relevancia de las fases de análisis, diseño e implementación.
Se busca que no sea una mera operación de copiar y pegar diagramas y extractos del código fuente, sino que realmente se justifiquen los caminos de solución que se han tomado, especialmente aquellos que no sean triviales.
Puede ser el lugar más adecuado para documentar los aspectos más interesantes del diseño y de la implementación, con un mayor hincapié en aspectos tales como el tipo de arquitectura elegido, los índices de las tablas de la base de datos, normalización y desnormalización, distribución en ficheros3, reglas de negocio dentro de las bases de datos (EDVHV GH GDWRV DFWLYDV), aspectos de desarrollo relacionados con el WWW...
Este apartado, debe convertirse en el resumen de la experiencia práctica del proyecto, y por sí mismo justifica que la memoria se convierta en un documento útil, fuente de referencia para los autores, los tutores y futuros alumnos.



\section{Análisis}


En la fase de análisis nos tenemos que asegurar de que se podrán cumplir los objetivos del proyecto: 

- Acceso a Moodle para extracción de datos
- Implementación y ejecución de nodos de forma aislada y en combinación con otros nodos de Knime, dentro de un workflow.




Fuente de datos real

- Explicar opciones de extracción de datos de Moodle. 
- Acotar el alcance de información a extraer de Moodle



Cursos de prueba

Posibilidad de incorporar fuentes de datos reales (asignatura)

Estudio de nodo de prueba y otros nodos similares (ejemplo paquete Twitter)

Programación en JAVA. 


Estudio de métodos de acceso a los datos de Moodle. En este punto se detecta que los servicios web de Moodle no ofrecen acceso a todos los
datos que se pueden visualizar desde la APP de Moodle. Aquí se puede hablar también de APP de Moodle y de acceso por Web Scraping. 





\section{Diseño}

- Opciones de reutilización de código: 
   UBUMonitor
   APP de Moodle 


- Análisis mediante Insomnia de los datos obtenidos a través de la API

- Organizar los sprints para abordar el proyecto de forma organizada



- Diseño de nodos. 

Se descubre que los nodos no se pueden ejecutar a nivel de código. Esto es, no podemos inyectar la funcionalidad completa de un nodo 
dentro de otro, ejecutándolo internamente. Knime ha impuesto esta restricción de diseño para evitar la alta dependencia entre nodos 
y los problemas que ello conllevaría a nivel de mantenimiento cuando un nodo queda obsoleto. 






\section{Implementación}



\subsection{Acceso a datos en Moodle}


Comentar aquí solución web services, limitaciones tanto de acceso a datos como de 

Finalmente se opta por web scraping utilizando la cuenta del profesor para acceder a los logs. 

Login con webservice de la APP de Moodle. Esto implica que la implementación de Moodle debe tener activa la opción de acceso de la APP. 



\subsection{Desarrollo de nodos en Knime}

Entorno de desarrollo:

Cómo montar el entorno de desarrollo según Knime
Nodo de muestra
Otros recursos (código de nodos desarrollados)
Añadir librerías (dependencias en proyecto knime)






Extensión: 

- KNIME Moodle Integration


Nodos desarrollados:

- Moodle Connector. 

Establece la conexión con una plataforma Moodle. Requiere una cuenta con perfil de profesor. 
Requiere que la plataforma Moodle tenga activado el acceso a la aplicación Móvil. 

Output Ports. 

0. A connection that can be used to access Moodle's API.

URL
usuario
contraseña


Salida: token

- Moodle Courses. 

core\_course\_get\_courses
core\_course\_get\_courses\_by\_field
core\_course\_search\_courses

Input Ports: 

0 Connection to Moodle's API (token)
1 Table containing... 

Output Ports: 

0 Información de cursos (ID cursos)



- Moodle Users. 

core\_enrol\_get\_enrolled\_users
anonymization


- Moodle Reports

logs
Live logs
Activity Reports
Overview Statistics
Course Participation
Activity completion
Statistics


- Reutilización de nodos desde programación 
Explicar la ventaja de poder utilizar nodos desde programación. Los nodos son clases que se pueden instanciar. 
La ventaja es que están pensados para trabajar con elementos (clases) de Knime. Así que esta estrategia será interesante solo cuando vayamos a trabajar con elementos de Knime,
como conversiones de tipos desde o hacia clases Knime. 

Explicar por qué no se ha podido utilizar esta estrategia.

https://forum.knime.com/t/using-node-without-gui/2044/6


Ejemplos de reutilización: 
 
 - JSON to table


Workflows desarrollados: 

Nota: los workflows compartidos son casi tan importantes como los nodos desarrollados, ya que permiten el uso directo de soluciones 

- Workflow básico (extraer logs)
- Workflow clasificación
- Workflow clustering





- Dependencias con proyectos externos

UBUMonitor
Versión 2.10.2 20220426
https://github.com/yjx0003/UBUMonitor/releases


JSON In Java (20220320)
https://github.com/stleary/JSON-java


org.knime.json 
Incluido en: 
https://update.knime.org/analytics-platform/UpdateSite_latest45.zip


¿ Moodle-REST-API-for-Java ? En principio parece que está más mantenido el UBUMonitor
https://github.com/yjx0003/Moodle-REST-API-for-Java







\section{Workflows}

Nodos externos utilizados en Workflows


https://hub.knime.com/redfield/extensions/se.redfield.arx.feature/latest