\apendice{Documentación técnica de programación}

\section{Introducción}

En este apéndice se describen los aspectos relevantes a nivel de programación para cualquier 
desarrollador interesado en continuar el proyecto o en desarrollar nuevas extensiones para la 
plataforma KNIME. 
\

\section{Estructura de directorios}

A continuación se presenta la estructura principal de archivos y carpetas de la extensión 
knime-moodle-integration desarrollada. No se han incluido las carpetas genéricas de definición 
del proyecto en Eclipse, que se pueden consultar directamente en el repositorio del proyecto. 

\begin{verbatim} 
\libraries -- Librerías externas (jar)
\src.org.knime.moodle.
  .internal  -- Librerías de uso interno
    .connection  -- Librería de conexión utilizada por el nodo Moodle Connector
      MoodleConnection.java
      MoodleConnectionPortObject.java
      MoodleConnectionPortObjectSpec.java
      MoodleLogin.Java
    .logs  -- Librería de Logs utilizada por el nodo Moodle Reports Logs
      Component.Java
      Component.java
      ComponentEvent.java
      Event.java
      LogDescription.java
      LogLine.java
      LogParameters.java
      Logs.java
      MoodleLogCreator.java
  .nodes  -- Nodos de la extensión
    .connector  -- Implementación del nodo Moodle Connector
      MoodleConnectorConfiguration.java
      MoodleConnectorNodeDialog.java
      MoodleConnectorNodeFactory.java
      MoodleConnectorNodeFactory.xml
      MoodleConnectorNodeModel.java
      MoodleConnectorNodePlugin.java
      MoodleConnectorNodeSettingsModel.java
      MoodleConnectorNodeView.java
      default.png
      moodle-connector.png
      package.html
    .courses  -- Implementación del nodo Moodle Courses
      MoodleCoursesNodeDialog.java
      MoodleCoursesNodeFactory.java
      MoodleCoursesNodeFactory.xml
      MoodleCoursesNodeModel.java
      MoodleCoursesNodeView.java
      default.png
      moodle-courses.png
      package.html
    .reports
      .grades  -- Implementación del nodo Moodle Reports Grades
        MoodleReportsGradesNodeDialog.java
        MoodleReportsGradesNodeFactory.java
        MoodleReportsGradesNodeFactory.xml
        MoodleReportsGradesNodeModel.java
        MoodleReportsGradesNodeView.java
        moodle-reports-grades.png
        package.html
      .logs  -- Implementación del nodo Moodle Reports Logs
        MoodleReportsLogsNodeDialog.java
        MoodleReportsLogsNodeFactory.java
        MoodleReportsLogsNodeFactory.xml
        MoodleReportsLogsNodeModel.java
        MoodleReportsLogsNodeView.java
        moodle-reports-logs.png
        package.html
    .users  -- Implementación del nodo Moodle Users
      MoodleUsersNodeDialog.java
      MoodleUsersNodeFactory.java
      MoodleUsersNodeFactory.xml
      MoodleUsersNodeModel.java
      MoodleUsersNodeView.java
      moodle-users.png
      package.html
\end{verbatim}



\subsection{Entorno de desarrollo con Eclipse y Knime SDK}

Los nodos de Knime se desarrollan en Java utilizando el IDE de programación Eclipse. Aunque desde la versión 4.6 también se permite
 la implementación de nodos totalmente programados en Python, en este proyecto los nodos se han implementado en Java para la versión KNIME 4.7.x. 
\

La documentación oficial de KNIME nos detalla paso a paso cómo preparar el entorno de trabajo necesario para desarrollar
 nuevos nodos en Java. Como primer paso debemos instalar Eclipse como IDE de desarrollo y KNIME SDK que nos proporciona las herramientas 
 necesarias de desarrollo para compilar nuestra extensión y ejecutarla en KNIME. Se pueden consultar las instrucciones detalladas en este enlace \url{https://github.com/knime/knime-sdk-setup}. 
\

Los pasos que tendremos que seguir son:

\begin{enumerate}
	\item Instalación de Java 17 (OpenJDK 17). Esta versión puede variar para futuras versiones de KNIME. 
	\item Instalar Eclipse. Debemos instalar la versión "Eclipse for RCP and RAP Developers 2022-06", que es la que se corresponde a KNIME 4.7.x. 
  Esta versión varía en función de la versión de KNIME en la que vayamos a trabajar. 
	\item Instalar Git. 
	\item Descargar Knime SDK y configurar eclipse. Siguiendo las instrucciones correspondientes, indicaremos cuál es la plataforma objetivo activa, de forma 
  que Eclipse pueda ejecutar KNIME y desplegar la extensión desarrollada. 
  \item Ejecutar KNIME desde KNIME Analytics Platform.launch -> Run as -> Knime Analytics Platform. 
\end{enumerate}  


\subsection{Instalación y compilación de la extensión KNIME Moodle Integration}

Por último, necesitamos incorporar la extensión knime-moodle-integration importando el proyecto desde el repositorio. 
\



\section{Desarrollo de nuevos nodos}

Las instrucciones previas nos sirven para montar un entorno de desarrollo para trabajar con la extensión desarrollada en este proyecto. Si se desea implementar una nueva extensión, 
se recomienda seguir las intrucciones que KNIME nos facilita mediante la implementación de un nodo de Ejemplo: 

\url{https://docs.knime.com/latest/analytics_platform_new_node_quickstart_guide/index.html#_introduction}


